\documentclass{bmcart}

%%%%%%%%%%%%%%%%%%%%%%%%%%%%%%%%%%%%%%%%%%%%%%
%%                                          %%
%% CARGA DE PAQUETES DE LATEX               %%
%%                                          %%
%%%%%%%%%%%%%%%%%%%%%%%%%%%%%%%%%%%%%%%%%%%%%%

%%% Load packages
\usepackage{amsthm,amsmath}
\usepackage{graphicx}
%\RequirePackage[numbers]{natbib}
\RequirePackage{hyperref}
\usepackage[utf8]{inputenc} %unicode support
%\usepackage[applemac]{inputenc} %applemac support if unicode package fails
%\usepackage[latin1]{inputenc} %UNIX support if unicode package fails


%%%%%%%%%%%%%%%%%%%%%%%%%%%%%%%%%%%%%%%%%%%%%%
%%                                          %%
%% COMIENZO DEL DOCUMENTO                   %%
%%                                          %%
%%%%%%%%%%%%%%%%%%%%%%%%%%%%%%%%%%%%%%%%%%%%%%

\begin{document}

	\begin{frontmatter}
	
		\begin{fmbox}
			\dochead{Research}
			
			%%%%%%%%%%%%%%%%%%%%%%%%%%%%%%%%%%%%%%%%%%%%%%
			%% INTRODUCIR TITULO PROYECTO               %%
			%%%%%%%%%%%%%%%%%%%%%%%%%%%%%%%%%%%%%%%%%%%%%%
			
			\title{Mitocondrias anormales en el tejido muscular}
			
			%%%%%%%%%%%%%%%%%%%%%%%%%%%%%%%%%%%%%%%%%%%%%%
			%% AUTORES. METER UNA ENTRADA AUTHOR        %%
			%% POR PERSONA                              %%
			%%%%%%%%%%%%%%%%%%%%%%%%%%%%%%%%%%%%%%%%%%%%%%
			
			\author[
			  addressref={aff1},                   % ESTA LINEA SE COPIA IGUAL PARA CADA AUTOR
			  corref={aff1},                       % ESTA LINEA SOLO DEBE TENERLA EL COORDINADOR DEL GRUPO
			  email={juliadelrio2003@uma.es}   % VUESTRO CORREO ACTIVO
			]{\inits{J.R.T.}\fnm{Julia} \snm{Del Río Toledo}} % inits: INICIALES DE AUTOR, fnm: NOMBRE DE AUTOR, snm: APELLIDOS DE AUTOR
			
			\author[
			  addressref={aff1},
			  email={javiimendeezz@uma.es}
			]{\inits{J.M.P.}\fnm{Javier} \snm{Méndez Parrilla}}
			
			%%%%%%%%%%%%%%%%%%%%%%%%%%%%%%%%%%%%%%%%%%%%%%
			%% AFILIACION. NO TOCAR                     %%
			%%%%%%%%%%%%%%%%%%%%%%%%%%%%%%%%%%%%%%%%%%%%%%
			
			\address[id=aff1]{%                           % unique id
			  \orgdiv{ETSI Informática},             % department, if any
			  \orgname{Universidad de Málaga},          % university, etc
			  \city{Málaga},                              % city
			  \cny{España}                                    % country
			}
		
		\end{fmbox}% comment this for two column layout
		
		\begin{abstractbox}
		
			\begin{abstract} % abstract
			
			%%%%%%%%%%%%%%%%%%%%%%%%%%%%%%%%%%%%%%%%%%%%%%%
			%% RESUMEN BREVE DE NO MAS DE 100 PALABRAS   %%
			%%%%%%%%%%%%%%%%%%%%%%%%%%%%%%%%%%%%%%%%%%%%%%%	
			
			\end{abstract}
			
			%%%%%%%%%%%%%%%%%%%%%%%%%%%%%%%%%%%%%%%%%%%%%%
			%% PALABRAS CLAVE DEL PROYECTO              %%
			%%%%%%%%%%%%%%%%%%%%%%%%%%%%%%%%%%%%%%%%%%%%%%
			
			\begin{keyword}
			\kwd{sample}
			\kwd{article}
			\kwd{author}
			\end{keyword}
		
		
		\end{abstractbox}
	
	\end{frontmatter}
	

	%%%%%%%%%%%%%%%%%%%%%%%%%%%%%%%%%
	%% COMIENZO DEL DOCUMENTO REAL %%
	%%%%%%%%%%%%%%%%%%%%%%%%%%%%%%%%%
	
	\section{Introducción}

Las mitocondrias, además de ser la principal fuente de ATP, desempeñan funciones esenciales en la regulación del metabolismo celular, incluyendo la catabolización de nutrientes y la gestión de la homeostasis redox \cite{Spinelli2018}. A lo largo de la evolución eucariota, su forma y comportamiento se adaptaron para garantizar la transmisión precisa de su genoma y responder a las demandas celulares \cite{Friedman2014}. En tejidos de alta demanda energética como el músculo esquelético, las mitocondrias juegan un papel crucial en la flexibilidad metabólica y la adaptación a estímulos como el ejercicio \cite{Memme2021, Smith2023}. La disfunción mitocondrial contribuye a diversas patologías \cite{QuintanaCabrera2023}, neurodegenerativas, cancerígenas \cite{Chan2020} o musculoesqueléticas \cite{Liu2017}.

	\section{Materiales y métodos}


\subsection{\textbf{Materiales}}

\subsubsection{Programas}
 

\textbf{R} es un lenguaje de programación y un entorno estadístico muy utilizado en bioinformática, análisis de datos y ciencia de datos.
Sirve para realizar análisis estadísticos complejos, visualización de datos y manipulación de datos biológicos, como análisis de expresión genética para lo que lo estamos utilizando en este caso.
Fue desarrollado por Robert Gentleman y Ross Ihaka en la Universidad de Auckland, Nueva Zelanda, y es un proyecto de código abierto mantenido por la comunidad R.\cite{jimenez2019introduccion}\\


\textbf{Python} es un lenguaje de programación, conocido por su simplicidad, versatilidad y eficiencia a la hora de programar.
Se utiliza en análisis de datos, desarrollo de modelos, análisis bioinformático, y en el manejo y procesamiento de grandes volúmenes de datos biológicos.
Fue creado por Guido van Rossum y ha crecido hasta convertirse en uno de los lenguajes más populares, con una comunidad activa que proporciona múltiples herramientas y bibliotecas.\cite{Python_Teams}
\subsubsection{Bases de Datos}


\textbf{HPO} (Human Phenotype Ontology) es una base de datos que proporciona una clasificación estándar de los términos fenotípicos humanos, cada uno con un identificador único.
En el contexto de la investigación genética, HPO permite relacionar los fenotipos observables ("mitocondrias anormales en el tejido muscular", identificado por HP:0008316) con genes y enfermedades específicas. Esto facilita el análisis y busca de datos genéticos en función de características fenotípicas.
Es mantenido por la comunidad científica, específicamente por el equipo de Monarch Initiative, para mejorar la precisión en los estudios de fenotipos y genética humana.\cite{kohler2014human}\\




\textbf{StringDB} es una base de datos que contiene información sobre interacciones proteína-proteína (PPI), recopilada de diversas fuentes como experimentos, bases de datos públicas y predicciones computacionales.
Esta base de datos permite investigar redes de proteínas interconectadas, facilitando el análisis de cómo las proteínas interactúan en diferentes contextos biológicos, algo esencial en estudios genéticos como los relacionados con HPO.
Desarrollada por un consorcio de investigadores y está disponible como un recurso en línea gratuito para la comunidad científica.\cite{szklarczyk2015string}\\

 

El \textbf{NCBI} (National Center for Biotechnology Information) es un centro de investigación en biotecnología de Estados Unidos que proporciona acceso a una vasta colección de bases de datos biológicas y genómicas.
Es utilizado para acceder a datos sobre genes, proteínas y secuencias genómicas, y proporciona herramientas de análisis. Nosotros lo utilizamos junto a HPO,para facilitar la correlación entre genes específicos y fenotipos.
Fue fundado por los Institutos Nacionales de Salud (NIH) de Estados Unidos y es un recurso fundamental para la investigación genética y bioinformática.\cite{jenuth1999ncbi}



\subsubsection{Bibliotecas}




\textbf{Igraph} es una biblioteca para el análisis y la visualización de redes complejas, desarrollada en múltiples lenguajes, incluyendo Python y R.
En bioinformática, se usa para construir y analizar redes de interacciones biológicas, como redes de proteínas o genes, ayudando a visualizar cómo las proteínas relacionadas con un fenotipo específico pueden interactuar entre sí.
Desarrollado por una comunidad internacional de programadores y científicos.\cite{valdez2016analisis}\\


\textbf{Pandas} es una biblioteca en Python que facilita la manipulación y el análisis de datos mediante estructuras de datos como DataFrames.
En estudios bioinformáticos, pandas se utiliza para manejar y procesar grandes volúmenes de datos, como los obtenidos de bases de datos genéticas. Es útil para organizar los datos de HPO y facilitar su análisis.
Fue creada por Wes McKinney y es ahora una biblioteca ampliamente adoptada en el ámbito de la ciencia de datos.\cite{mckinney2011pandas}\\


\textbf{Requests} es una biblioteca de Python que permite realizar solicitudes HTTP de forma sencilla.
Es utilizada para acceder a APIs y extraer datos de bases de datos en línea, como HPO o StringDB, permitiendo automatizar la obtención de datos relevantes para el estudio de fenotipos y genes.
Fue desarrollada por Kenneth Reitz y es una de las bibliotecas más populares para trabajar con APIs en Python.\cite{chandra2015python}\\


\textbf{Networkx} es una biblioteca en Python para la creación, manipulación y análisis de redes complejas.
Se utiliza para modelar y analizar redes de interacciones genéticas o de proteínas. Por ejemplo, en el análisis de datos de StringDB, networkx permite estudiar las relaciones entre proteínas de forma visual y cuantitativa.
Es una biblioteca de código abierto mantenida por la comunidad, utilizada ampliamente en estudios de redes biológicas y sociales.\cite{hagberg2020networkx}\\


\textbf{Matplotlib.pyplot} es una biblioteca de Python que permite crear gráficos y visualizaciones de datos de forma detallada y personalizada.
Lo utilizamos para representar datos biológicos y resultados de análisis de manera visual, lo cual facilita la interpretación de datos complejos, como los patrones de interacción entre proteínas o genes relacionados con un fenotipo específico.
Fue desarrollada por John D. Hunter y es mantenida por la comunidad científica y de datos.\cite{ari2014matplotlib}



\subsection{\textbf{Métodos}}


\subsubsection{\textbf{Obtención de datos y construcción de la red de interacciones}}

El análisis comenzó con la recopilación de genes asociados nuestro fenotipo (HP:0008316 (mitocondrias anormales en tejido muscular)). Desarrollamos un script en Python que interactuaba con la API de la Ontología del Fenotipo Humano (\textbf{HPO}) para extraer dichos genes. El script utilizaba la biblioteca \textit{requests} para realizar solicitudes GET a la API y guardaba la lista de genes en un archivo de texto.

La lista obtenida se procesó para analizar sus interacciones proteicas utilizando la API de \textbf{STRINGdb}. Se definieron los siguientes parámetros para la solicitud:

\begin{itemize}
	\item Especie: Homo sapiens (código NCBI: 9606).
	\item Umbral de confianza: 0.8 para incluir solo las interacciones de alta fiabilidad.
	\item Tipo de red: \textit{confidence} para enfocarse en interacciones respaldadas por evidencia experimental y predictiva.
\end{itemize}

La respuesta de la API se convirtió en un DataFrame usando \textit{pandas} y se construyó un grafo con \textit{NetworkX}, donde los nodos representaban genes y las aristas, las interacciones proteicas. Las aristas fueron ponderadas de acuerdo a la puntuación de confianza.

La visualización del grafo se realizó con \textit{matplotlib}, mostrando la red de interacciones y etiquetando las aristas con sus puntuajes de interacción.

\subsubsection{\textbf{Análisis de la red y visualización avanzada}}

El análisis detallado de la red se llevó a cabo en \textbf{R} usando la biblioteca \textbf{iGraph}. En primer lugar, se importaron los datos correspondientes a los nodos y aristas de la red desde archivos CSV generados previamente a partir del análisis con STRINGdb. Estos datos incluían información sobre las conexiones entre los genes y las características asociadas a cada nodo.

Una vez cargados, se generó un grafo dirigido utilizando la función \textit{graph\_from\_data\_frame()}, que permitía establecer la red con la dirección de las interacciones y asignar atributos a los nodos y aristas. Este grafo fue visualizado con diversas configuraciones para explorar las características de la red de forma más profunda. Se realizaron múltiples pruebas de visualización, ajustando parámetros como el tamaño y el color de los nodos, de manera que reflejaran atributos específicos, tales como el grado de conectividad o la pertenencia a una comunidad.

El análisis de conectividad fue un paso clave en la evaluación de la red, ya que permitió identificar los genes con mayor número de conexiones, conocidos como “hubs”. Estos genes tienen una relevancia significativa en la red biológica, ya que suelen estar asociados con funciones críticas y pueden actuar como puntos de control en los procesos celulares. Para visualizar la distribución del grado de los nodos, se construyeron histogramas que mostraban cómo se distribuían las conexiones en la red y se creó un mapa de calor que representaba el peso de las aristas, proporcionando una vista global de la intensidad de las interacciones.

La representación gráfica de la red se llevó a cabo utilizando la función \textit{plot()} de iGraph, ajustando parámetros visuales como el tamaño de los nodos, la curvatura de las aristas y la disposición de los nombres de los genes en el gráfico. En algunos casos, se emplearon paletas de colores específicas para resaltar comunidades de genes, utilizando el método de detección de comunidades basado en algoritmos como la propagación de etiquetas. Este enfoque permitió no solo visualizar la red, sino también identificar subgrupos de genes con patrones de interacción similares, lo que facilitó la comprensión de posibles agrupaciones funcionales en la red de interacciones proteicas.

\subsubsection{\textbf{Análisis de enriquecimiento funcional}}

El análisis de enriquecimiento funcional se realizó en Python utilizando herramientas para determinar los procesos biológicos y las vías metabólicas sobrerrepresentadas en los genes de la red. Los resultados se exportaron a archivos CSV y se visualizaron mediante gráficos para facilitar la interpretación.







	
\section{Resultados}


En el marco de este estudio, se obtuvieron inicialmente 48 genes asociados con el fenotipo mitocondrial desde la base de datos Human Phenotype Ontology (HPO). Estos genes se relacionaron con funciones celulares esenciales en la mitocondria, como la cadena respiratoria, la biogénesis mitocondrial y la regulación del metabolismo energético. Los genes fueron seleccionados con el objetivo de explorar su posible interrelación dentro de una red de interacciones de proteínas, utilizando la base de datos STRING.

Al realizar el análisis de interacciones utilizando los identificadores de estos 48 genes, se observó que solo 38 genes interaccionaron entre sí dentro de la red generada. Este hallazgo sugirió que, aunque todos los genes descargados estaban asociados con el fenotipo mitocondrial, no todos ellos presentaron interacciones directas en el contexto de las proteínas mitocondriales. En particular, algunos genes no mostraron evidencia de interacción directa con otros genes seleccionados, lo que podría haber indicado que sus roles en la biología mitocondrial no dependían de interacciones proteicas directas o que interactuaban a través de mecanismos más complejos no reflejados en esta red.

Uno de los genes más relevantes en este análisis fue MYH7, que estaba relacionado con la miocardiopatía y era conocido por su papel en la función mitocondrial. Sin embargo, MYH7 no presentó interacciones con otros genes mitocondriales en esta red, lo que sugirió que su influencia en el fenotipo mitocondrial podría haber estado mediada por mecanismos independientes, tal vez a través de la regulación de procesos mitocondriales o por interacciones con otras proteínas no incluidas en la red de interacciones analizada.


Para profundizar en el análisis de la red obtenida, se han aplicado diferentes algoritmos de clustering con el objetivo de identificar comunidades o grupos de proteínas que interactúen más estrechamente entre sí. Los algoritmos utilizados fueron: Fast Greedy, Edge Betweenness, Walktrap, Infomap y Label Propagation, y se evaluaron mediante la métrica de modularidad, que mide la calidad de las particiones realizadas.
\vspace{1em}

\begin{tabular}{|l|c|r|}
	\hline
	\textbf{Algoritmo} & \textbf{Modularidad} & \textbf{Número de Clusters} \\
	\hline
	Fast Greedy & 0.06917908 & 3 \\
	\hline
	Walktrap clustering & 0.02946764 & 9 \\
	\hline
	Edge Betweenness clustering & 0.01066141 & 1 \\
	\hline
	Infomap clustering & \(2.220446 \times 10^{-16}\) & 1 \\
	\hline
	Label\_Propagation clustering & \(2.220446 \times 10^{-16}\) & 1 \\
	\hline
\end{tabular}


\vspace{1em}
	

Entre los algoritmos evaluados, se observó que Fast Greedy destacó como el algoritmo más efectivo, lo que indicó que logró separar de forma significativa las diferentes comunidades.

El algoritmo identificó tres clúster principales:

En el primer clúster se encontraron nodos como NDUFS7, NDUFAF1, NDUFB10, NDUFS6, NDUFV2. Este clúster reunió muchas proteínas de la familia NDUF, fundamentales para la producción de energía en la célula. Entre estos elementos se observó una alta conectividad.

El segundo clúster contuvo como nodos más destacados MT-CO1, MT-ATP6, MT-ND6, MT-ND5. Este grupo pareció estar más enfocado en la respiración celular y la fosforilación oxidativa.

El tercer clúster estuvo representado por elementos menos conectados, como por ejemplo: NDUFS2, TIMMDC1, NUBPL. Este clúster podría haber indicado que se estaban enfocando en procesos relacionados con la cadena de transporte de electrones.

El dendrograma generado mediante clustering jerárquico aportó una perspectiva adicional sobre cómo se relacionaban los genes seleccionados. Esta herramienta organizó los genes en una estructura jerárquica basada en sus similitudes funcionales, permitiendo observar con mayor claridad la proximidad o separación entre ellos dentro de la red de interacciones.

\paragraph{Agrupación de genes} \mbox{}\\

En las ramas inferiores del dendrograma, destacaron genes como \textbf{NDUFS7}, \textbf{NDUFS3} y \textbf{NDUFS6}, los cuales presentaban una alta conectividad en la red de interacciones y se agrupaban estrechamente. Esto reforzó los hallazgos de clustering realizados previamente, donde estos genes se identificaron como parte de comunidades definidas por algoritmos como Fast Greedy. Su proximidad sugirió roles interrelacionados en la producción de energía celular.

\paragraph{Separación de grupos} \mbox{}\\

A medida que se ascendía en la jerarquía, genes como \textbf{NDUFB10} y \textbf{MT-CO1} comenzaban a divergir, formando grupos más distantes. Este patrón era consistente con los resultados del algoritmo Walktrap, donde dichos genes pertenecían a clusters separados, lo que indicaba funciones menos directamente conectadas dentro de la red.

\paragraph{Genes periféricos} \mbox{}\\

En las ramas más alejadas del dendrograma se encontraban genes como \textbf{NUBPL} y \textbf{TIMMDC1}, los cuales presentaban menor conectividad con los clusters principales. Sin embargo, estos genes podrían desempeñar roles complementarios, interactuando de manera más indirecta con genes centrales como \textbf{MT-ATP6}.

\paragraph{Relación entre clusters} \mbox{}\\

La estructura jerárquica observada en el dendrograma confirmó la organización de los clusters identificados previamente. Los genes del primer y segundo cluster mostraron una conexión más estrecha, con roles críticos en la cadena respiratoria y la fosforilación oxidativa. Por otro lado, los genes del tercer cluster aparecieron más tempranamente segregados, sugiriendo funciones complementarias o especializadas dentro del contexto mitocondrial.

\paragraph{}

Tras haber observado la separación en comunidades mediante el uso de diferentes algoritmos de clustering, se procedió a realizar un análisis funcional detallado para cada uno de los grupos identificados. Este análisis tuvo como objetivo explorar las funciones biológicas principales asociadas a los genes de cada comunidad, así como su posible implicación en procesos mitocondriales clave. Adicionalmente, se analizaron los nodos principales (\textbf{top nodes}) de la red, definidos como aquellos genes con mayor conectividad y centralidad, que podrían desempeñar roles esenciales dentro de la red biológica.

\paragraph{Análisis funcional del primer clúster} \mbox{}\\

El primer clúster, que agrupa principalmente proteínas de la familia NDUF (e.g., NDUFS7, NDUFAF1, NDUFB10), mostró un enriquecimiento significativo en procesos relacionados con la organización mitocondrial y el ensamblaje del complejo I de la cadena respiratoria. Entre los términos más destacados se encuentran "NADH dehydrogenase complex assembly" y "mitochondrial respiratory chain complex assembly". Estos resultados refuerzan el papel fundamental de este grupo en la producción de energía celular mediante la fosforilación oxidativa.

El alto \textbf{GeneRatio} (\(\geq 0.6\)) y los valores ajustados de significancia extremadamente bajos (\( 5.6 \times 10^{-40} \)) sugieren que la mayoría de los genes de este clúster están funcionalmente relacionados. Esto subraya la importancia de estos genes en el mantenimiento de la bioenergética mitocondrial y sugiere posibles implicaciones en enfermedades asociadas con disfunción mitocondrial.

\paragraph{Análisis funcional del segundo clúster} \mbox{}\\

El segundo clúster, compuesto por genes como MT-CO1, MT-ATP6, MT-ND6 y MT-ND5, presentó una clara asociación con procesos relacionados con la respiración celular y la síntesis de ATP. Los términos enriquecidos, como "proton motive force-driven ATP synthesis" y "aerobic electron transport chain", indican una fuerte implicación en la generación de energía mediante la fosforilación oxidativa.

Con un \textbf{GeneRatio} cercano a 0.84 y valores de p.adjust (\( 1.5 \times 10^{-10} \)), este clúster se posiciona como central en el metabolismo energético. Además, la inclusión de términos asociados a la biosíntesis de nucleótidos sugiere un rol clave en procesos que demandan altos niveles de energía celular.

\paragraph{Análisis funcional del tercer clúster} \mbox{}\\

El tercer clúster, que incluye genes como NDUFS2, TIMMDC1 y NUBPL, mostró un menor grado de conectividad, lo que podría indicar funciones más especializadas o complementarias. Sin embargo, su enriquecimiento en términos como "oxidative phosphorylation" y "mitochondrial electron transport, NADH to ubiquinone" refuerza su papel en la cadena de transporte de electrones.

Aunque el \textbf{GeneRatio} es ligeramente menor en comparación con los dos clústeres anteriores, los valores de p.adjust (\( 1.1 \times 10^{-49} \)) confirman la relevancia de los procesos asociados. Estos resultados sugieren que los genes de este clúster contribuyen a la funcionalidad mitocondrial de manera más indirecta, pero igualmente esencial.

\paragraph{Nodos principales (Top Nodes)} \mbox{}\\

Los top nodes de la red corresponden a genes con alta conectividad y centralidad, lo que los posiciona como elementos clave dentro de la red biológica. Entre los procesos destacados asociados a estos nodos se encuentran "ATP biosynthetic process" y "aerobic electron transport chain". El \textbf{GeneRatio} extremadamente alto (~1.0) refleja que casi todos los genes analizados participan en estas funciones centrales.

Estos nodos no solo actúan como puntos de integración funcional, sino que también representan posibles dianas terapéuticas en condiciones patológicas asociadas con la disfunción mitocondrial. Los valores de significancia ajustada (\( 2 \times 10^{-9} \)) validan la robustez de este análisis, destacando la relevancia biológica de los genes más centrales en la red.

\paragraph{Lista de genes} \mbox{}\\

El análisis funcional de la lista completa de genes seleccionados muestra un enriquecimiento significativo en procesos clave como "mitochondrion organization", "oxidative phosphorylation" y "mitochondrial respiratory chain complex assembly". Estos resultados refuerzan la relación de los genes con la regulación de la bioenergética mitocondrial y el ensamblaje de componentes esenciales.

El \textbf{GeneRatio} alto (\(\geq 0.6\)) y los valores de p.adjust extremadamente bajos (\( 1.1 \times 10^{-49} \)) validan la importancia de estos procesos. Además, se observa una concordancia con los hallazgos de los clústeres, destacando genes clave como MT-ATP6 y miembros de la familia NDUF.

Este análisis complementa los resultados anteriores, ofreciendo una visión global de la relevancia funcional de los genes seleccionados en los procesos mitocondriales fundamentales.

\paragraph{}

El análisis funcional de los clústeres identificados por el algoritmo Fast Greedy confirma que los genes agrupados comparten funciones biológicas específicas relacionadas con la mitocondria, principalmente en la producción de energía y el ensamblaje de componentes mitocondriales. Los genes del primer y segundo clúster mostraron una conexión más estrecha, mientras que los del tercer clúster aparecen más segregados, con funciones posiblemente complementarias.

El análisis de los top nodes aporta una perspectiva adicional, destacando genes clave que actúan como nodos centrales en la red. Este enfoque combinado entre análisis funcional y estructural permite no solo entender las funciones principales de cada clúster, sino también identificar posibles reguladores centrales de la red de interacciones. Esto refuerza la utilidad de los análisis de clustering y funcionales como herramientas para explorar la complejidad biológica de los sistemas mitocondriales.

	\section{Discusión}

El presente estudio proporciona una visión integral sobre los mecanismos moleculares asociados al fenotipo de mitocondrias anormales en tejido muscular, destacando los procesos biológicos clave que subyacen a la disfunción mitocondrial y su relación con la salud muscular. A través del análisis de redes proteicas e integrando un enfoque de enriquecimiento funcional, hemos identificado una serie de rutas y mecanismos implicados en la biogénesis mitocondrial, la producción de ATP y la regulación metabólica, fundamentales para el funcionamiento muscular adecuado. \cite{Rebeca}

Los resultados obtenidos revelan que los genes involucrados en el fenotipo de mitocondrias anormales se agrupan en tres clústeres funcionales, cada uno con un papel esencial en la homeostasis mitocondrial y la bioenergética celular. El clúster 1, centrado en la organización mitocondrial y el ensamblaje de la cadena respiratoria, confirma que los procesos de biogénesis mitocondrial son cruciales para la estabilidad estructural y funcional de las mitocondrias. Este hallazgo es consistente con investigaciones previas que sugieren que alteraciones en la formación y mantenimiento de los complejos respiratorios mitocondriales están asociadas con patologías musculares y enfermedades relacionadas con la energía, como la sarcopenia. \cite{Milner}

Por otro lado, el clúster 2 destaca procesos relacionados con la síntesis de ATP, impulsada por gradientes de protones, y el transporte electrónico acoplado, lo que refuerza el papel central de las mitocondrias en la producción de energía para el músculo esquelético. La fosforilación oxidativa, que es clave en la producción de ATP, se encuentra fuertemente representada en este clúster, lo que concuerda con estudios previos que han vinculado disfunciones mitocondriales con la pérdida de fuerza muscular y la fatiga en diversas patologías.

El clúster 3, relacionado también con la fosforilación oxidativa, proporciona una mayor evidencia del papel de las mitocondrias en la regulación de la energía celular \cite{Zhao2011} . Los procesos involucrados en este clúster sugieren que los genes asociados con este fenotipo podrían estar regulando no solo la generación de energía, sino también la eficiencia energética del músculo, lo que tiene implicaciones significativas para la comprensión de la fatiga muscular y la debilidad en enfermedades mitocondriales.

Además, el análisis de enriquecimiento funcional refuerza la relevancia de estos hallazgos al identificar rutas biológicas asociadas con procesos clave como la organización mitocondrial y la cadena respiratoria, subrayando la conexión entre los genes analizados y su función en la bioenergética celular \cite{Cameron2011}. Los valores de GeneRatio altos y los valores de p.adjust extremadamente bajos destacan la robustez estadística de los resultados, lo que sugiere que los procesos identificados son esenciales y están estrechamente vinculados con la disfunción mitocondrial en el tejido muscular.

El análisis de nodos principales identificó genes altamente conectados, como MT-ATP6, que actúan como reguladores clave en procesos esenciales como la síntesis de ATP. Su alta centralidad en la red biológica los posiciona como posibles dianas terapéuticas para abordar patologías relacionadas con la disfunción mitocondrial, como la sarcopenia y enfermedades neuromusculares.
Sin embargo, algunas limitaciones inherentes al análisis, como la falta de interacciones observadas en ciertos genes (por ejemplo, MYH7), sugieren que estos genes pueden participar en vías biológicas alternativas o estar involucrados en mecanismos reguladores indirectos no capturados por el análisis de red. Estos hallazgos destacan la necesidad de complementar los enfoques computacionales con estudios experimentales que validen las interacciones propuestas y examinen los mecanismos de regulación no detectados en este tipo de análisis.

En resumen, este estudio contribuye significativamente a la comprensión de los mecanismos moleculares asociados con las mitocondrias anormales en tejido muscular, proporcionando una base sólida para futuras investigaciones. La identificación de rutas clave y procesos involucrados en la bioenergética mitocondrial ofrece nuevas perspectivas para el desarrollo de estrategias terapéuticas dirigidas a mitigar las disfunciones mitocondriales y sus efectos sobre la salud muscular.
 Estos hallazgos no solo abren nuevas oportunidades para el desarrollo de estrategias terapéuticas, sino que también sugieren enfoques para mitigar los efectos de la disfunción mitocondrial en patologías musculares.
	\section{Conclusiones}

Este estudio ha logrado identificar y analizar las vías y procesos biológicos asociados a la patología HP:0008316 (mitocondrias anormales en el tejido muscular) mediante el uso de biología de sistemas. A través de la construcción de redes biológicas, se identificaron clústeres funcionales clave relacionados con la biogénesis mitocondrial, la fosforilación oxidativa y la producción de ATP.

Los resultados confirman la implicación de estos procesos en la disfunción mitocondrial y su relación con enfermedades musculares, proporcionando una base para futuras investigaciones que podrían llevar al desarrollo de estrategias terapéuticas dirigidas a restaurar la funcionalidad mitocondrial.







	
	
	%%%%%%%%%%%%%%%%%%%%%%%%%%%%%%%%%%%%%%%%%%%%%%
	%% OTRA INFORMACIÓN                         %%
	%%%%%%%%%%%%%%%%%%%%%%%%%%%%%%%%%%%%%%%%%%%%%%
	
	\begin{backmatter}
	
		\section*{Abreviaciones}%% if any
			Indicar lista de abreviaciones mostrando cada acrónimo a que corresponde
		
		\section*{Disponibilidad de datos y materiales}%% if any
			\href{https://github.com/juliadelrio0/project_template}{Enlace al proyecto en GitHub}
			
		
		\section*{Contribución de los autores}
			Usando las iniciales que habéis definido al comienzo del documento, debeis indicar la contribución al proyecto en el estilo:
			J.E : Encargado del análisis de coexpresión con R, escritura de resultados; J.R.S : modelado de red con python y automatizado del código, escritura de métodos; ...
			OJO: que sea realista con los registros que hay en vuestros repositorios de github. 
		
		
		%%%%%%%%%%%%%%%%%%%%%%%%%%%%%%%%%%%%%%%%%%%%%%%%%%%%%%%%%%%%%%%%%%%%%%%%%%%%%%%%%%%%%%%%
		%% BIBLIOGRAFIA: no teneis que tocar nada, solo sustituir el archivo bibliography.bib %%
		%% por el que hayais generado vosotros                                                %%
		%%%%%%%%%%%%%%%%%%%%%%%%%%%%%%%%%%%%%%%%%%%%%%%%%%%%%%%%%%%%%%%%%%%%%%%%%%%%%%%%%%%%%%%%
		
		\bibliographystyle{bmc-mathphys} % Style BST file (bmc-mathphys, vancouver, spbasic).
		\bibliography{bibliography}      % Bibliography file (usually '*.bib' )
	
	\end{backmatter}
\end{document}
