\section{Materiales y métodos}


\subsection{\textbf{Materiales}}

\subsubsection{Programas}
 

\textbf{R} es un lenguaje de programación y un entorno estadístico muy utilizado en bioinformática, análisis de datos y ciencia de datos.
Sirve para realizar análisis estadísticos complejos, visualización de datos y manipulación de datos biológicos, como análisis de expresión genética para lo que lo estamos utilizando en este caso.
Fue desarrollado por Robert Gentleman y Ross Ihaka en la Universidad de Auckland, Nueva Zelanda, y es un proyecto de código abierto mantenido por la comunidad R.\\


\textbf{Python} es un lenguaje de programación, conocido por su simplicidad, versatilidad y eficiencia a la hora de programar.
Se utiliza en análisis de datos, desarrollo de modelos, análisis bioinformático, y en el manejo y procesamiento de grandes volúmenes de datos biológicos.
Fue creado por Guido van Rossum y ha crecido hasta convertirse en uno de los lenguajes más populares, con una comunidad activa que proporciona múltiples herramientas y bibliotecas.
\subsubsection{Bases de Datos}


\textbf{HPO} (Human Phenotype Ontology) es una base de datos que proporciona una clasificación estándar de los términos fenotípicos humanos, cada uno con un identificador único.
En el contexto de la investigación genética, HPO permite relacionar los fenotipos observables ("mitocondrias anormales en el tejido muscular", identificado por HP:0008316) con genes y enfermedades específicas. Esto facilita el análisis y busca de datos genéticos en función de características fenotípicas.
Es mantenido por la comunidad científica, específicamente por el equipo de Monarch Initiative, para mejorar la precisión en los estudios de fenotipos y genética humana.\\




\textbf{StringDB} es una base de datos que contiene información sobre interacciones proteína-proteína (PPI), recopilada de diversas fuentes como experimentos, bases de datos públicas y predicciones computacionales.
Esta base de datos permite investigar redes de proteínas interconectadas, facilitando el análisis de cómo las proteínas interactúan en diferentes contextos biológicos, algo esencial en estudios genéticos como los relacionados con HPO.
Desarrollada por un consorcio de investigadores y está disponible como un recurso en línea gratuito para la comunidad científica.\\

 

El \textbf{NCBI} (National Center for Biotechnology Information) es un centro de investigación en biotecnología de Estados Unidos que proporciona acceso a una vasta colección de bases de datos biológicas y genómicas.
Es utilizado para acceder a datos sobre genes, proteínas y secuencias genómicas, y proporciona herramientas de análisis. Nosotros lo utilizamos junto a HPO,para facilitar la correlación entre genes específicos y fenotipos.
Fue fundado por los Institutos Nacionales de Salud (NIH) de Estados Unidos y es un recurso fundamental para la investigación genética y bioinformática.



\subsubsection{Bibliotecas}




\textbf{Igraph} es una biblioteca para el análisis y la visualización de redes complejas, desarrollada en múltiples lenguajes, incluyendo Python y R.
En bioinformática, se usa para construir y analizar redes de interacciones biológicas, como redes de proteínas o genes, ayudando a visualizar cómo las proteínas relacionadas con un fenotipo específico pueden interactuar entre sí.
Desarrollado por una comunidad internacional de programadores y científicos.\\


\textbf{Pandas} es una biblioteca en Python que facilita la manipulación y el análisis de datos mediante estructuras de datos como DataFrames.
En estudios bioinformáticos, pandas se utiliza para manejar y procesar grandes volúmenes de datos, como los obtenidos de bases de datos genéticas. Es útil para organizar los datos de HPO y facilitar su análisis.
Fue creada por Wes McKinney y es ahora una biblioteca ampliamente adoptada en el ámbito de la ciencia de datos.\\


\textbf{Requests} es una biblioteca de Python que permite realizar solicitudes HTTP de forma sencilla.
Es utilizada para acceder a APIs y extraer datos de bases de datos en línea, como HPO o StringDB, permitiendo automatizar la obtención de datos relevantes para el estudio de fenotipos y genes.
Fue desarrollada por Kenneth Reitz y es una de las bibliotecas más populares para trabajar con APIs en Python.\\


\textbf{Networkx} es una biblioteca en Python para la creación, manipulación y análisis de redes complejas.
Se utiliza para modelar y analizar redes de interacciones genéticas o de proteínas. Por ejemplo, en el análisis de datos de StringDB, networkx permite estudiar las relaciones entre proteínas de forma visual y cuantitativa.
Es una biblioteca de código abierto mantenida por la comunidad, utilizada ampliamente en estudios de redes biológicas y sociales.\\


\textbf{Matplotlib.pyplot} es una biblioteca de Python que permite crear gráficos y visualizaciones de datos de forma detallada y personalizada.
Lo utilizamos para representar datos biológicos y resultados de análisis de manera visual, lo cual facilita la interpretación de datos complejos, como los patrones de interacción entre proteínas o genes relacionados con un fenotipo específico.
Fue desarrollada por John D. Hunter y es mantenida por la comunidad científica y de datos.



\subsection{\textbf{Métodos}}


\subsubsection{\textbf{Obtención de datos y construcción de la red de interacciones}}

El análisis comenzó con la recopilación de genes asociados nuestro fenotipo (HP:0008316 (mitocondrias anormales en tejido muscular)). Desarrollamos un script en Python que interactuaba con la API de la Ontología del Fenotipo Humano (\textbf{HPO}) para extraer dichos genes. El script utilizaba la biblioteca \textit{requests} para realizar solicitudes GET a la API y guardaba la lista de genes en un archivo de texto.

La lista obtenida se procesó para analizar sus interacciones proteicas utilizando la API de \textbf{STRINGdb}. Se definieron los siguientes parámetros para la solicitud:

\begin{itemize}
	\item Especie: Homo sapiens (código NCBI: 9606).
	\item Umbral de confianza: 0.8 para incluir solo las interacciones de alta fiabilidad.
	\item Tipo de red: \textit{confidence} para enfocarse en interacciones respaldadas por evidencia experimental y predictiva.
\end{itemize}

La respuesta de la API se convirtió en un DataFrame usando \textit{pandas} y se construyó un grafo con \textit{NetworkX}, donde los nodos representaban genes y las aristas, las interacciones proteicas. Las aristas fueron ponderadas de acuerdo a la puntuación de confianza.

La visualización del grafo se realizó con \textit{matplotlib}, mostrando la red de interacciones y etiquetando las aristas con sus puntuajes de interacción.

\subsubsection{\textbf{Análisis de la red y visualización avanzada}}

El análisis detallado de la red se llevó a cabo en \textbf{R} usando la biblioteca \textbf{iGraph}. En primer lugar, se importaron los datos correspondientes a los nodos y aristas de la red desde archivos CSV generados previamente a partir del análisis con STRINGdb. Estos datos incluían información sobre las conexiones entre los genes y las características asociadas a cada nodo.

Una vez cargados, se generó un grafo dirigido utilizando la función \textit{graph\_from\_data\_frame()}, que permitía establecer la red con la dirección de las interacciones y asignar atributos a los nodos y aristas. Este grafo fue visualizado con diversas configuraciones para explorar las características de la red de forma más profunda. Se realizaron múltiples pruebas de visualización, ajustando parámetros como el tamaño y el color de los nodos, de manera que reflejaran atributos específicos, tales como el grado de conectividad o la pertenencia a una comunidad.

El análisis de conectividad fue un paso clave en la evaluación de la red, ya que permitió identificar los genes con mayor número de conexiones, conocidos como “hubs”. Estos genes tienen una relevancia significativa en la red biológica, ya que suelen estar asociados con funciones críticas y pueden actuar como puntos de control en los procesos celulares. Para visualizar la distribución del grado de los nodos, se construyeron histogramas que mostraban cómo se distribuían las conexiones en la red y se creó un mapa de calor que representaba el peso de las aristas, proporcionando una vista global de la intensidad de las interacciones.

La representación gráfica de la red se llevó a cabo utilizando la función \textit{plot()} de iGraph, ajustando parámetros visuales como el tamaño de los nodos, la curvatura de las aristas y la disposición de los nombres de los genes en el gráfico. En algunos casos, se emplearon paletas de colores específicas para resaltar comunidades de genes, utilizando el método de detección de comunidades basado en algoritmos como la propagación de etiquetas. Este enfoque permitió no solo visualizar la red, sino también identificar subgrupos de genes con patrones de interacción similares, lo que facilitó la comprensión de posibles agrupaciones funcionales en la red de interacciones proteicas.

\subsubsection{\textbf{Análisis de enriquecimiento funcional}}

El análisis de enriquecimiento funcional se realizó en Python utilizando herramientas para determinar los procesos biológicos y las vías metabólicas sobrerrepresentadas en los genes de la red. Los resultados se exportaron a archivos CSV y se visualizaron mediante gráficos para facilitar la interpretación.






