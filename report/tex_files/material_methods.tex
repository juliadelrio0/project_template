\section{Materiales y métodos}


\subsection{\textbf{Materiales}}

\subsubsection{Programas}
 

\textbf{R} es un lenguaje de programación y un entorno estadístico muy utilizado en bioinformática, análisis de datos y ciencia de datos.
Sirve para realizar análisis estadísticos complejos, visualización de datos y manipulación de datos biológicos, como análisis de expresión genética para lo que lo estamos utilizando en este caso.
Fue desarrollado por Robert Gentleman y Ross Ihaka en la Universidad de Auckland, Nueva Zelanda, y es un proyecto de código abierto mantenido por la comunidad R.\cite{jimenez2019introduccion}\\


\textbf{Python} es un lenguaje de programación, conocido por su simplicidad, versatilidad y eficiencia a la hora de programar.
Se utiliza en análisis de datos, desarrollo de modelos, análisis bioinformático, y en el manejo y procesamiento de grandes volúmenes de datos biológicos.
Fue creado por Guido van Rossum y ha crecido hasta convertirse en uno de los lenguajes más populares, con una comunidad activa que proporciona múltiples herramientas y bibliotecas.\cite{Python_Teams}
\subsubsection{Bases de Datos}


\textbf{HPO} (Human Phenotype Ontology) es una base de datos que proporciona una clasificación estándar de los términos fenotípicos humanos, cada uno con un identificador único.
En el contexto de la investigación genética, HPO permite relacionar los fenotipos observables ("mitocondrias anormales en el tejido muscular", identificado por HP:0008316) con genes y enfermedades específicas. Esto facilita el análisis y busca de datos genéticos en función de características fenotípicas.
Es mantenido por la comunidad científica, específicamente por el equipo de Monarch Initiative, para mejorar la precisión en los estudios de fenotipos y genética humana.\cite{kohler2014human}\\




\textbf{StringDB} es una base de datos que contiene información sobre interacciones proteína-proteína (PPI), recopilada de diversas fuentes como experimentos, bases de datos públicas y predicciones computacionales.
Esta base de datos permite investigar redes de proteínas interconectadas, facilitando el análisis de cómo las proteínas interactúan en diferentes contextos biológicos, algo esencial en estudios genéticos como los relacionados con HPO.
Desarrollada por un consorcio de investigadores y está disponible como un recurso en línea gratuito para la comunidad científica.\cite{szklarczyk2015string}\\

 

El \textbf{NCBI} (National Center for Biotechnology Information) es un centro de investigación en biotecnología de Estados Unidos que proporciona acceso a una vasta colección de bases de datos biológicas y genómicas.
Es utilizado para acceder a datos sobre genes, proteínas y secuencias genómicas, y proporciona herramientas de análisis. Nosotros lo utilizamos junto a HPO,para facilitar la correlación entre genes específicos y fenotipos.
Fue fundado por los Institutos Nacionales de Salud (NIH) de Estados Unidos y es un recurso fundamental para la investigación genética y bioinformática.\cite{jenuth1999ncbi}



\subsubsection{Bibliotecas}




\textbf{Igraph} es una biblioteca para el análisis y la visualización de redes complejas, desarrollada en múltiples lenguajes, incluyendo Python y R.
En bioinformática, se usa para construir y analizar redes de interacciones biológicas, como redes de proteínas o genes, ayudando a visualizar cómo las proteínas relacionadas con un fenotipo específico pueden interactuar entre sí.
Desarrollado por una comunidad internacional de programadores y científicos.\cite{valdez2016analisis}\\


\textbf{Pandas} es una biblioteca en Python que facilita la manipulación y el análisis de datos mediante estructuras de datos como DataFrames.
En estudios bioinformáticos, pandas se utiliza para manejar y procesar grandes volúmenes de datos, como los obtenidos de bases de datos genéticas. Es útil para organizar los datos de HPO y facilitar su análisis.
Fue creada por Wes McKinney y es ahora una biblioteca ampliamente adoptada en el ámbito de la ciencia de datos.\cite{mckinney2011pandas}\\


\textbf{Requests} es una biblioteca de Python que permite realizar solicitudes HTTP de forma sencilla.
Es utilizada para acceder a APIs y extraer datos de bases de datos en línea, como HPO o StringDB, permitiendo automatizar la obtención de datos relevantes para el estudio de fenotipos y genes.
Fue desarrollada por Kenneth Reitz y es una de las bibliotecas más populares para trabajar con APIs en Python.\cite{chandra2015python}\\


\textbf{Networkx} es una biblioteca en Python para la creación, manipulación y análisis de redes complejas.
Se utiliza para modelar y analizar redes de interacciones genéticas o de proteínas. Por ejemplo, en el análisis de datos de StringDB, networkx permite estudiar las relaciones entre proteínas de forma visual y cuantitativa.
Es una biblioteca de código abierto mantenida por la comunidad, utilizada ampliamente en estudios de redes biológicas y sociales.\cite{hagberg2020networkx}\\


\textbf{Matplotlib.pyplot} es una biblioteca de Python que permite crear gráficos y visualizaciones de datos de forma detallada y personalizada.
Lo utilizamos para representar datos biológicos y resultados de análisis de manera visual, lo cual facilita la interpretación de datos complejos, como los patrones de interacción entre proteínas o genes relacionados con un fenotipo específico.
Fue desarrollada por John D. Hunter y es mantenida por la comunidad científica y de datos.\cite{ari2014matplotlib}



\subsection{\textbf{Métodos}}

\subsubsection{\textbf{Obtención de datos y construcción de la red de interacciones}}

El análisis comenzó con la identificación de genes asociados al fenotipo de interés, definido por el código HP:0008316, que corresponde a "mitocondrias anormales en tejido muscular" en la Ontología del Fenotipo Humano (HPO, por sus siglas en inglés). Este paso fue crucial para establecer una base de datos inicial que sustentara el estudio.

Para llevar a cabo este análisis de manera sistemática, desarrollamos un script en Python llamado \textbf{\textit{hpo\_genes\_fetcher.py}} que interactuaba con la API de HPO. La elección de Python como lenguaje principal se basó en su flexibilidad y en su amplio ecosistema de bibliotecas, como \textit{requests} y \textit{pandas}, que son ideales para manejar datos y realizar solicitudes con las APIs. El script se diseñó para automatizar la extracción de genes relacionados con el fenotipo, lo que evitaba errores manuales y garantizaba la reproducibilidad del proceso.

Se utilizó la biblioteca \textit{requests} para realizar solicitudes HTTP de tipo GET a los endpoints específicos de la API de HPO. El identificador del fenotipo, HP:0008316, se incluyó como parámetro en la URL de la solicitud, asegurando que los datos recuperados estuvieran específicamente relacionados con este término fenotípico. La respuesta de la API, en formato JSON (estándar ampliamente utilizado para el intercambio de datos estructurados en aplicaciones web), contenía una lista dentro del campo "genes", que incluía los nombres de los genes asociados al fenotipo. Este formato estructurado fue procesado utilizando la biblioteca \textit{pandas}, lo que permitió organizar la información en un DataFrame, una estructura de datos tabular que facilita su análisis y manipulación.

Una vez organizados, los datos se depuraron para eliminar duplicados o entradas incompletas, con lo que se garantizó que la lista de genes obtenida estuviera lista para su uso en análisis posteriores. Finalmente, los genes se almacenaron en un archivo de texto estructurado para facilitar su acceso y uso en las etapas siguientes del estudio.

La lista de genes recopilada fue utilizada como entrada en un segundo script de Python, \textbf{\textit{fetch\_interactions.py}}, que se encargó de analizar las interacciones proteicas entre los genes utilizando la API de \textbf{STRINGdb}. Este análisis es crucial para comprender cómo interactúan entre sí las proteínas codificadas por los genes, lo que puede revelar información sobre sus funciones en el fenotipo.

Utilizando dicha API, el script envió solicitudes POST que incluían los nombres de los genes formateados con \%0d como delimitador, cumpliendo con los requisitos específicos de la API. Se configuraron parámetros como la especie de interés (Homo sapiens, código NCBI: 9606) y un umbral de confianza de 0.5, significando esto que las interacciones incluidas tienen al menos un 50\% de probabilidad de ser funcionalmente relevantes, según los cálculos de STRINGdb.

La respuesta de la API, en formato JSON, proporcionó información detallada sobre las interacciones, incluyendo los genes conectados, la naturaleza de las interacciones y una puntuación de confianza. Además, el script incluyó una función para descargar y guardar una representación visual de la red directamente desde STRINGdb en formato PNG, lo que facilitó la documentación gráfica de las interacciones proteicas.

\subsubsection{\textbf{Análisis de la red y visualización avanzada}}

El análisis detallado de la red de interacciones proteicas se realizó en \textbf{R} utilizando la biblioteca \textbf{iGraph}. Este paso fue fundamental para profundizar en las propiedades estructurales y funcionales de la red, permitiendo explorar relaciones clave entre los genes asociados a nuestro fenotipo. Para comenzar, los datos generados previamente con STRINGdb, que contenían la lista de nodos y aristas, fueron exportados en formato CSV y posteriormente importados en R mediante las funciones estándar \textit{read.csv()}. Este enfoque aseguró la interoperabilidad entre Python y R, facilitando la integración del flujo de trabajo.

En R, los datos se transformaron en un grafo dirigido utilizando la función \textit{graph\_from\_data\_frame()}. Este grafo representaba los genes como nodos y las interacciones entre ellos como aristas ponderadas, donde el peso correspondía a la puntuación de confianza calculada por STRINGdb. La dirección de aristas permitió representar el flujo funcional entre los genes.

La visualización inicial del grafo se realizó utilizando la función \textit{plot()} de iGraph, con personalizaciones que resaltaban características clave de la red. Se ajustaron colores y tamaños de los nodos según su grado de conectividad, lo que permitió identificar rápidamente los genes más conectados o "hubs". Para optimizar la interpretación visual, se probaron diferentes algoritmos de disposición, como \textit{layout\_with\_fr} (disposición de Fruchterman-Reingold) y \textit{layout\_nicely}, que distribuyen los nodos de manera equilibrada según sus conexiones. Estas configuraciones facilitaron la identificación de patrones estructurales en la red y permitieron una representación gráfica intuitiva de las interacciones proteicas.

Además del grado de los nodos, se calcularon otras métricas topológicas relevantes, como la centralidad de intermediación, que identifica genes clave para el flujo de información, y el coeficiente de agrupamiento, que mide la tendencia de los genes a formar clústeres densos.

La distribución del grado de los nodos se realizó mediante histogramas, que mostraban cómo se distribuían las conexiones entre los genes. Adicionalmente, se generaron mapas de calor que representaban la intensidad de las interacciones (pesos de las aristas), proporcionando una visión global de las relaciones más relevantes. Estas visualizaciones se realizaron utilizando bibliotecas de R como \textit{ggplot2} y \textit{heatmap()}, que facilitaron la interpretación de los datos.

Finalmente, se evaluaron las comunidades detectadas mediante algoritmos de agrupamiento. Cada comunidad fue analizada en términos de su relevancia funcional en el contexto del fenotipo, considerando la conectividad interna y los genes principales en cada módulo. Este enfoque permitió identificar subgrupos funcionales de genes, lo que ayudó a interpretar mejor las relaciones biológicas subyacentes.

\subsubsection{\textbf{Análisis de enriquecimiento funcional}}






