
\section{Resultados}


En el marco de este estudio, se obtuvieron inicialmente 48 genes asociados con el fenotipo mitocondrial desde la base de datos Human Phenotype Ontology (HPO), que se relacionan con funciones celulares esenciales en la mitocondria, como la cadena respiratoria, la biogénesis mitocondrial, y la regulación del metabolismo energético. Estos genes fueron seleccionados con el objetivo de explorar su posible interrelación dentro de una red de interacciones de proteínas, utilizando la base de datos STRING.

Al realizar el análisis de interacciones utilizando los identificadores de estos 48 genes, se observó que solo 38 genes interaccionaron entre sí dentro de la red generada. Este hallazgo sugiere que, aunque todos los genes descargados están asociados con el fenotipo mitocondrial, no todos ellos tienen interacciones directas en el contexto de las proteínas mitocondriales. En particular, algunos genes no mostraron evidencia de interacción directa con otros genes seleccionados, lo que podría indicar que sus roles en la biología mitocondrial pueden no depender de interacciones proteicas directas o que interactúan a través de mecanismos más complejos que no son reflejados en esta red.

Uno de los genes más relevantes en este análisis es MYH7, que está relacionado con la miocardiopatía y es conocido por su papel en la función mitocondrial. Sin embargo, MYH7 no presentó interacciones con otros genes mitocondriales en esta red, lo que sugiere que su influencia en el fenotipo mitocondrial podría estar mediada por mecanismos independientes, tal vez a través de la regulación de procesos mitocondriales o por interacciones con otras proteínas no presentes en la red de interacciones utilizadas en este análisis.