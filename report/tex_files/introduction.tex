
\section{Introducción}

Las mitocondrias, además de ser la principal fuente de ATP, desempeñan funciones esenciales en la regulación del metabolismo celular, incluyendo la catabolización de nutrientes y la gestión de la homeostasis redox \cite{Spinelli2018}. A lo largo de la evolución eucariota, su forma y comportamiento se adaptaron para garantizar la transmisión precisa de su genoma y responder a las demandas celulares \cite{Friedman2014}. En tejidos de alta demanda energética como el músculo esquelético, las mitocondrias juegan un papel crucial en la flexibilidad metabólica y la adaptación a estímulos como el ejercicio \cite{Memme2021, Smith2023}. La disfunción mitocondrial contribuye a diversas patologías \cite{QuintanaCabrera2023}, neurodegenerativas, cancerígenas \cite{Chan2020} o musculoesqueléticas \cite{Liu2017}.


La mitofagia es el proceso mediante el cual las mitocondrias dañadas se eliminan selectivamente. Los defectos en la mitofagia pueden llevar a la acumulación de mitocondrias disfuncionales, afectando aún más la salud muscular. \cite{Leudec}
La disfunción mitocondrial \cite{chen} en estos tejidos se ha asociado con varias patologías relacionadas con la edad, incluidas la insuficiencia cardíaca y la sarcopenia \cite{Boengler} .La reducción de la función mitocondrial en el músculo esquelético contribuye a la pérdida de masa muscular y fuerza, lo que afecta significativamente la movilidad y la calidad de vida.
Otro factor determinante que provocan las mitocondrias disfuncionales es el aumento de ROS (especies reactivas de oxígeno), provocando daño oxidativo en músculos y otros tejidos. \cite{java}

Además, la disrupción en la dinámica mitocondrial (fusión y fisión) contribuye a la atrofia muscular, con un aumento de mitocondrias hinchadas y fragmentadas. Estas alteraciones son comunes en enfermedades como la distrofia muscular, donde la pérdida de proteínas estructurales agrava los efectos negativos de la disfunción mitocondrial en el tejido muscular.

