\section{Introducción}

Las mitocondrias, además de ser la principal fuente de ATP, desempeñan funciones esenciales en la regulación del metabolismo celular, incluyendo la catabolización de nutrientes y la gestión de la homeostasis redox . A lo largo de la evolución eucariota, su forma y comportamiento se adaptaron para garantizar la transmisión precisa de su genoma y responder a las demandas celulares . En tejidos de alta demanda energética como el músculo esquelético, las mitocondrias juegan un papel crucial en la flexibilidad metabólica y la adaptación a estímulos como el ejercicio. La disfunción mitocondrial contribuye a diversas patologías celulares, neurodegenerativas o musculoesqueléticas. 
