\section{Discusión}

El presente estudio proporciona una visión integral sobre los mecanismos moleculares asociados al fenotipo de mitocondrias anormales en tejido muscular, destacando los procesos biológicos clave que subyacen a la disfunción mitocondrial y su relación con la salud muscular. A través del análisis de redes proteicas e integrando un enfoque de enriquecimiento funcional, hemos identificado una serie de rutas y mecanismos implicados en la biogénesis mitocondrial, la producción de ATP y la regulación metabólica, fundamentales para el funcionamiento muscular adecuado. \cite{Rebeca}

Los resultados obtenidos revelan que los genes involucrados en el fenotipo de mitocondrias anormales se agrupan en tres clústeres funcionales, cada uno con un papel esencial en la homeostasis mitocondrial y la bioenergética celular. El clúster 1, centrado en la organización mitocondrial y el ensamblaje de la cadena respiratoria, confirma que los procesos de biogénesis mitocondrial son cruciales para la estabilidad estructural y funcional de las mitocondrias. Este hallazgo es consistente con investigaciones previas que sugieren que alteraciones en la formación y mantenimiento de los complejos respiratorios mitocondriales están asociadas con patologías musculares y enfermedades relacionadas con la energía, como la sarcopenia. \cite{Milner}

Por otro lado, el clúster 2 destaca procesos relacionados con la síntesis de ATP, impulsada por gradientes de protones, y el transporte electrónico acoplado, lo que refuerza el papel central de las mitocondrias en la producción de energía para el músculo esquelético. La fosforilación oxidativa, que es clave en la producción de ATP, se encuentra fuertemente representada en este clúster, lo que concuerda con estudios previos que han vinculado disfunciones mitocondriales con la pérdida de fuerza muscular y la fatiga en diversas patologías.

El clúster 3, relacionado también con la fosforilación oxidativa, proporciona una mayor evidencia del papel de las mitocondrias en la regulación de la energía celular \cite{Zhao2011} . Los procesos involucrados en este clúster sugieren que los genes asociados con este fenotipo podrían estar regulando no solo la generación de energía, sino también la eficiencia energética del músculo, lo que tiene implicaciones significativas para la comprensión de la fatiga muscular y la debilidad en enfermedades mitocondriales.

Además, el análisis de enriquecimiento funcional refuerza la relevancia de estos hallazgos al identificar rutas biológicas asociadas con procesos clave como la organización mitocondrial y la cadena respiratoria, subrayando la conexión entre los genes analizados y su función en la bioenergética celular \cite{Cameron2011}. Los valores de GeneRatio altos y los valores de p.adjust extremadamente bajos destacan la robustez estadística de los resultados, lo que sugiere que los procesos identificados son esenciales y están estrechamente vinculados con la disfunción mitocondrial en el tejido muscular.

El análisis de nodos principales identificó genes altamente conectados, como MT-ATP6, que actúan como reguladores clave en procesos esenciales como la síntesis de ATP. Su alta centralidad en la red biológica los posiciona como posibles dianas terapéuticas para abordar patologías relacionadas con la disfunción mitocondrial, como la sarcopenia y enfermedades neuromusculares.
Sin embargo, algunas limitaciones inherentes al análisis, como la falta de interacciones observadas en ciertos genes (por ejemplo, MYH7), sugieren que estos genes pueden participar en vías biológicas alternativas o estar involucrados en mecanismos reguladores indirectos no capturados por el análisis de red. Estos hallazgos destacan la necesidad de complementar los enfoques computacionales con estudios experimentales que validen las interacciones propuestas y examinen los mecanismos de regulación no detectados en este tipo de análisis.

En resumen, este estudio contribuye significativamente a la comprensión de los mecanismos moleculares asociados con las mitocondrias anormales en tejido muscular, proporcionando una base sólida para futuras investigaciones. La identificación de rutas clave y procesos involucrados en la bioenergética mitocondrial ofrece nuevas perspectivas para el desarrollo de estrategias terapéuticas dirigidas a mitigar las disfunciones mitocondriales y sus efectos sobre la salud muscular.
 Estos hallazgos no solo abren nuevas oportunidades para el desarrollo de estrategias terapéuticas, sino que también sugieren enfoques para mitigar los efectos de la disfunción mitocondrial en patologías musculares.